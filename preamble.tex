\usepackage{import}
\usepackage[
  % colorlinks,
  % linkcolor=blue,
  % Include section numbers in pdf bookmarks
  bookmarksnumbered=true
]{hyperref}

\usepackage{tipa}

\usepackage{tikz}
\usetikzlibrary{positioning}
\usepackage{tikz-dependency}

\usepackage{xspace}

\usepackage{csquotes}

% fancier tables
\usepackage{booktabs}

% advanced table column specifications (<{}, >{}, m{})
\usepackage{array}

\usepackage{multirow}

% dashed lines in tables with \hdashline, \cdashline
\usepackage{arydshln}

\usepackage[
  % natbib=true,
  backend=biber,
  style=authoryear,
  % sorting=nty,
  seconds=true,
  alldates=iso,
  citetracker=true,
]{biblatex}

% better automatic ref handling (supports capitalizing ref names)
\usepackage[nameinlink,noabbrev]{cleveref}

% Subfigures
% \usepackage{caption}
\usepackage{subcaption}

% caption containing a source (most likely \cite)
\newcommand\captionsource[2]{\caption[#1]{#1 {\small (from #2)}}}

% \usepackage[acronym, toc]{glossaries}
\newcommand\gls[1]{#1\xspace}

\newcommand\ti\textipa
% an example word
\newcommand\ex[1]{\emph{#1}}
% the CVCV / strict-CV model
\newcommand\CVCV{CVCV\xspace}
% Phonological context (#RT, TT#, ...)
\newcommand\ctx[1]{#1}

% =============================================================
% word structure drawing commands
% =============================================================
\newcounter{vowel}
\newcounter{consonant}

\newenvironment{structure}{
  \setcounter{vowel}{0}
  \setcounter{consonant}{0}
  %
  \newcommand\drawCV[1]{% param: count
    \foreach \x in {0,1,...,##1} {
        \node (C\x) at (1.5 * \x       , 0) {$C_{\x}$};
        \node (V\x) at (1.5 * \x + 0.75, 0) {$V_{\x}$};
    }
  }
  %
  \newcommand\relation[3]{% params: Text, Coord1, Coord2
    \draw[->] (##2.north) ++(-0.1,0) -- ++(0,.2) -- node[above]{##1} (##3 |-, .5) -- (##3);
  }
  \newcommand\norelation[3]{% params: Text, Coord1, Coord2
    \draw[->] (##2.north) ++(-0.1,0) -- ++(0,.2) -- node[above]{##1} node{\tiny\textbf{/}} (##3 |-, .5) -- (##3);
  }
  \newcommand\Gov[2]{\relation{Gvt}{##1}{##2}}
  \newcommand\NoGov[2]{\norelation{Gvt}{##1}{##2}}
  \newcommand\Lic[2]{\relation{Lic}{##1}{##2}}
  \newcommand\NoLic[2]{\norelation{Lic}{##1}{##2}}
  %
  \newcommand\V[2][]{% params: [style] vowel
    \edef\pos{V\thevowel}
    \node[##1] (p\pos) at (\pos |-, -1) {##2};
    \draw[##1] (\pos) -- (p\pos);
    \stepcounter{vowel}
  }
  \newcommand\C[2][]{% params: [style] consonant
    \edef\pos{C\theconsonant}
    \node[##1] (p\pos) at (\pos |-, -1) {##2};
    \draw[##1] (\pos) -- (p\pos);
    \stepcounter{consonant}
  }
  \newcommand\emptyV{
    \stepcounter{vowel}
  }
  %
  \newcommand\longV[1]{% param: vowel
    \edef\firstpos{V\thevowel}
    \stepcounter{vowel}
    \stepcounter{consonant}
    \edef\secondpos{V\thevowel}
    \V{##1}
    \draw[dashed] (p\secondpos) -- (\firstpos);
    \relation{Lic}{\secondpos}{\firstpos}
  }
  \newcommand\wordstart{
    \node (word-start) at (1.5*\theconsonant + 0.75/2, -1) {\#};
    \draw (C\theconsonant) -- (word-start) -- (V\thevowel);
    \stepcounter{consonant}
    \stepcounter{vowel}
  }
  \newcommand\fen{
    \edef\pos{V\thevowel}
    \draw[<-] (\pos.north) -- ++(0,0.2) -- node[above]{Gov} ++(1,0);
  }
  \par\noindent%
  \begin{tikzpicture}[
    floating/.style={draw opacity=0}
  ]
}{
  \end{tikzpicture}
  \par\noindent%
}

% COMMANDS TO BE USED DURING EDITING

\usepackage{marginnote}
\renewcommand*{\marginfont}{\color{red}\sffamily\scriptsize}
% print marginnote on the left
\newcommand\leftmarginnote[1]{\reversemarginpar\marginnote{#1}\normalmarginpar}
\newcommand\rawmarker[1][2pt]{\rule{#1}{9pt}}
% a simple red box, used to mark positions in the text
\newcommand\marker{\textcolor{red}{\rawmarker}\xspace}
\newcommand\lmarknote[1]{\leftmarginnote{#1}\marker}
\newcommand\marknote[1]{\marginnote{#1}\marker}
\newcommand\marginmarker[1][2pt]{\marginnote{\rawmarker[#1]}}

\newcommand\wordunsure[1]{\textcolor{red}{#1\marginmarker}\xspace}

% should be translated later
\newcommand\tr[1]{\textcolor{orange}{#1\marginmarker}\xspace}
% comma?
\newcommand\co{\textbf{\color{orange},\marginmarker[1pt]}\xspace}
\newcommand\TODO[1]{{\color{red}\bfseries\setlength{\fboxrule}{3pt}\fbox{TODO} #1}}
\newenvironment{deleted}{\begingroup\color{gray}}{\endgroup}
% annotation (structure, ..)
\newcommand\annot[1]{\textcolor{purple}{#1\marginmarker}\xspace}