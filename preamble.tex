\usepackage{etoolbox}

\newtoggle{fastcompile}
% \toggletrue{fastcompile}

% FIXME: just for now
\usepackage[top=0.3in, left=0.7in, right=1.2in]{geometry}

\usepackage{import}
\usepackage[
  % colorlinks,
  % linkcolor=blue,
  % Include section numbers in pdf bookmarks
  bookmarksnumbered=true
]{hyperref}

\usepackage{tipa}

\usepackage{tikz}
\usetikzlibrary{positioning}
\usetikzlibrary{tikzmark}
\usepackage{tikz-dependency}

\usepackage{xspace}

\usepackage{csquotes}

% fancier tables
\usepackage{booktabs}

% advanced table column specifications (<{}, >{}, m{})
\usepackage{array}

\usepackage{multirow}

% dashed lines in tables with \hdashline, \cdashline
\usepackage{arydshln}

\iftoggle{fastcompile}{
  \newcommand\draftnote{(draft $\Rightarrow$ empty)}
  % general
  \renewcommand\tableofcontents{\section*{Contents \draftnote}}
  % biblatex
  \newcommand\addbibresource[1]{}
  \newcommand{\parencite}[2][]{(#1 #2)}
  \newcommand\printbibliography{\section*{Bibliography \draftnote}}
  % glossaries
  \newcommand\gls[1]{#1}
  \newcommand\glspl[1]{#1}
  % cleverref
  \NewDocumentCommand{\cref}{v}{#1}
  \NewDocumentCommand{\Cref}{v}{#1}
}{
  \usepackage[
    % natbib=true,
    backend=biber,
    style=authoryear,
    % sorting=nty,
    seconds=true,
    alldates=iso,
    citetracker=true,
  ]{biblatex}

  \usepackage[acronym, toc]{glossaries}
  %!TEX root = main-text.tex
%==============================================================================

% \newacr[options]{short}{long}
\newcommand\newacr[3][]{\newacronym[description={#3},#1]{#2}{#2}{#3}}
% \newacrpl[options]{short}{long}{shortplural}{longplural}
\newcommand\newacrpl[5][]{\newacronym[description={#3},shortplural={#4},longplural={#5}]{#2}{#2}{#3}}

\newacr{IG}{Infrasegmental Government}
\newacrpl{FEN}{Final Empty Nucleus}{FEN}{Final Empty Nuclei}

  % better automatic ref handling (supports capitalizing ref names)
  \usepackage[nameinlink,noabbrev]{cleveref}
}

% more control over enumerate and itemize environments
\usepackage{enumitem}

% Subfigures
\usepackage{caption}
\usepackage{subcaption}

% caption containing a source (most likely \cite)
\newcommand\captionsource[2]{\caption[#1]{#1 {\small (from #2)}}}

\newcommand\ti\textipa
% an example word
\newcommand\ex[1]{\emph{#1}}
% the CVCV / strict-CV model
\newcommand\CVCV{CVCV\xspace}
% Standard Government Phonology
\newcommand\SGP{Standard Government Phonology\xspace}
% Phonological context (#RT, TT#, ...)
\newcommand\ctx[1]{#1}
% Rule/derivation: #1 -> #2 / #3
\newcommand\deriv[3]{#1 $\to$ #2 \quad\slash{}\quad #3}
% Schwa mobile
\newcommand\schwaMobi{\textsc{\small E}\xspace}
% Schwa constans
\newcommand\schwaCons{\text\textschwa\xspace}

% Allow using \to outside of math mode as well
\renewcommand\to{\ensuremath{\rightarrow}}
\renewcommand\implies{\ensuremath{\Rightarrow}}


% =============================================================
% Conditional compilation
% =============================================================
\ifdef{\included}{\ErrorIncludedWasAlreadyDefined}{}
\newcommand\includethese[1]{\forcsvlist{\listadd\included}{#1}}
\newcommand\includeconditionally[2]{\ifinlist{#1}{\included}{#2}{}}
\renewcommand\includeonly[1]{\xdef\included{}\listxadd\included{#1}}


% =============================================================
% Definitions
% =============================================================
\usepackage{amsthm}
% \theoremstyle{definition}
\newtheorem*{definition}{Definition}

% \newenvironment{definition}[2]{%
%   \unskip(#1) #2
%   \begin{quotation}%
%   \noindent%
% }{    
%   \end{quotation}
% }

% shortcut command for text quotes
\newcommand\q[3][]{\textquote[{\cite[#1]{#2}}]{#3}}


% =============================================================
% phonological features, rules, derivations
% =============================================================
\newcommand\feat[1]{%
  \ensuremath{\left[%
    \renewcommand{\arraystretch}{1}%
    \begin{tabular}{@{}c@{}}#1\end{tabular}%
  \right]}%
}

\newcommand\phongroup[2][c]{%
  \scalebox{0.5}[1]{\ensuremath{\displaystyle%
    \left\{\scalebox{2}[1]{%
      \renewcommand{\arraystretch}{1.5}%
      \begin{tabular}{@{}#1@{}}#2\end{tabular}%
    }\right\}%
  }}%
}

\newcommand\phongroupi[2][c]{%
  \scalebox{0.5}[1]{\ensuremath{%
    \left\{\scalebox{2}[1]{%
      \renewcommand{\arraystretch}{1.5}%
      \begin{tabular}{@{}#1@{}}#2\end{tabular}%
    }\right\}%
  }}%
}

\newcommand\phonctx{\rule[-1.5pt]{4ex}{.5pt}}

% \newlength{\mylength}
% \settowidth{\mylength}{+}
% \showthe\mylength


% =============================================================
% word structure drawing commands
% =============================================================
\newcounter{vowel}
\newcounter{consonant}

% \usetikzlibrary{backgrounds}
% \pgfsetlayers{background,main}

\newenvironment{structure}[2][1.5]{ % args: [node_dist] subtext
  \setcounter{vowel}{1}
  \setcounter{consonant}{1}
  %
  \newcommand\drawCV[1]{% param: count
  \numdef\xEnd{##1 + 1}
    \foreach \x in {1,...,\xEnd} {
      \node[transparent] (C\x) at (#1 * \x           , 0) {$C_{\x}$};
      \node[transparent] (V\x) at (#1 * \x + 0.5 * #1, 0) {$V_{\x}$};
    }
  }
  \newcommand\drawVCV[1]{% param: count
  \node[transparent] (V1) at (#1 * 1 + 0.5 * #1, 0) {$V_1$};
  \numdef\xEnd{##1 + 1}
    \foreach \x in {2,...,\xEnd} {
      \node[transparent] (C\x) at (#1 * \x           , 0) {$C_{\x}$};
      \node[transparent] (V\x) at (#1 * \x + 0.5 * #1, 0) {$V_{\x}$};
    }
    \stepcounter{consonant}
  }
  %
  \newcommand\relation[3]{% params: Text, Coord1, Coord2
    \draw[->] (##2.north) ++(-0.1,0) -- ++(0,.2) -- node[above]{##1} (##3 |-, .5) -- (##3);
  }
  \newcommand\norelation[3]{% params: Text, Coord1, Coord2
    \draw[->] (##2.north) ++(-0.1,0) -- ++(0,.2) -- node[above]{##1} node{\tiny\textbf{/}} (##3 |-, .5) -- (##3);
  }
  \newcommand\relationO[3]{% params: Text, Coord1, Coord2
    \draw[->] (##2.north) -- ++(0,.8) -- node[above]{##1} (##3 |-, 1.1) -- (##3.north);
  }
  \newcommand\Gov[3][]{
    \ifstrequal{##1}{over}
      {\relationO{Gvt}{##2}{##3}}
      {\relation{Gvt}{##2}{##3}}
  }
  \newcommand\NoGov[2]{\norelation{Gvt}{##1}{##2}}
  \newcommand\Lic[2]{\relation{Lic}{##1}{##2}}
  \newcommand\NoLic[2]{\norelation{Lic}{##1}{##2}}
  %
  \newcommand\structureset[1]{\pgfkeys{/structure/.cd,##1}}
  \newcommand\StructureHandleSkeletonNode[2]{ % internal, args: {C|V} style
    \ifstrequal{##1}{V}{
      \let\currcounter=\thevowel
      \numdef\nextvowel{\value{vowel} + 1}
    }{
      \let\currcounter=\theconsonant
      \edef\nextvowel{\thevowel}
    }
    \edef\pos{##1\currcounter}
    % set default first, then (possibly) overwrite with args
    \structureset{
      % name (subscript to C/V node)
      n/.initial={\currcounter},
      n/.get=\skeletonnodename,
      n/.store in=\skeletonnodename,
      % gov relation
      gov/.code={
        \ifstrequal{####1}{true}{\Gov{V\nextvowel}{\pos}}{}
        \ifstrequal{####1}{false}{\NoGov{V\nextvowel}{\pos}}{}
      },
      gov/.default=true,
      nogov/.style={gov=false},
      % lic relation
      lic/.code={
        \ifstrequal{####1}{true}{\Lic{V\nextvowel}{\pos}}{}
        \ifstrequal{####1}{false}{\NoLic{V\nextvowel}{\pos}}{}
      },
      lic/.default=true,
      nolic/.style={lic=false},
      % floating melodies
      floating/.style={/tikz/draw opacity=0},
      ##2
    }
    \node at (\pos) {$##1_\text{\skeletonnodename}$};
  }
  %
  \newcommand\V[2][]{% params: [style] vowel
    \edef\pos{V\thevowel}
    \StructureHandleSkeletonNode{V}{##1}
    \node[anchor=base] (p\pos) at (\pos |-, -1) {##2};
    \draw (\pos) -- (p\pos);
    \stepcounter{vowel}
  }
  \newcommand\C[2][]{% params: [style] consonant
    \edef\pos{C\theconsonant}
    \StructureHandleSkeletonNode{C}{##1}
    \node[anchor=base] (p\pos) at (\pos |-, -1) {##2};
    \draw (\pos) -- (p\pos);
    \stepcounter{consonant}
  }
  \newcommand\emptyV[1][]{ % optionally pass 'gov' or 'nogov'
    \edef\prevpos{V\thevowel}
    \StructureHandleSkeletonNode{V}{##1}
    \stepcounter{vowel}
  }
  \newcommand\emptyC{
    \StructureHandleSkeletonNode{C}{}
    \stepcounter{consonant}
  }
  %
  \newcommand\longV[1]{% param: vowel
    \edef\firstpos{V\thevowel}
    \StructureHandleSkeletonNode{V}{}
    \StructureHandleSkeletonNode{C}{}
    \stepcounter{vowel}
    \stepcounter{consonant}
    \edef\secondpos{V\thevowel}
    \V{##1}
    \draw[dashed] (p\secondpos) -- (\firstpos);
    \relation{Lic}{\secondpos}{\firstpos}
  }
  \newcommand\wordstart{
    \node (C0) at (0       , 0) {$C_0$};
    \node (V0) at (0.5 * #1, 0) {$V_0$};
    \node[anchor=base] (word-start) at (0.25 * #1, -1) {\#};
    \draw (C0) -- (word-start) -- (V0);
    \Gov{V1}{V0}
  }
  \newcommand\fen[1][]{ % optionally pass 'domain' for domain-final (but not word-final) FEN
    \edef\pos{V\thevowel}
    \ifstrequal{##1}{domain}{
      % \draw[dashed] (\pos.north) ++(.25,0.1) -- ++(0,-1.6);
      \def\shiftRight{0}
    }{
      \StructureHandleSkeletonNode{V}{}
      \def\shiftRight{0.8}
    }
    \draw[<-] (\pos.north) -- ++(0,0.2) -- node[above]{Gvt} ++(\shiftRight,0);
  }
  \newcommand\IG[2]{
    \draw[->,double] (p##1) -- node[below]{IG} (p##2);
  }
  \newcommand\Ng[1][]{ % optionally pass 'lic' or 'nolic'
    \edef\firstpos{C\theconsonant}
    \C{N}
    \emptyV
    \edef\secondpos{C\theconsonant}
    \C[##1]{g}
    \draw (p\secondpos.north) -- (\firstpos.south);
  }
  \par\noindent%
  \begin{tikzpicture}[
    % framed,
    baseline=(V1.base),
    execute at end picture={\ifblank{#2}{}{
      \node[below=.5ex, align=flush center] at (current bounding box.south) {#2};
    }},
  ]
}{
  \end{tikzpicture}
  \par\noindent%
}

% COMMANDS TO BE USED DURING EDITING

\usepackage{marginnote}
\renewcommand*{\marginfont}{\color{red}\sffamily\scriptsize}
% print marginnote on the left
\newcommand\leftmarginnote[1]{\reversemarginpar\marginnote{#1}\normalmarginpar}
\newcommand\rawmarker[1][2pt]{\rule{#1}{9pt}}
% a simple red box, used to mark positions in the text
\newcommand\marker{\textcolor{red}{\rawmarker}\xspace}
\newcommand\lmarknote[1]{\leftmarginnote{#1}\marker}
\newcommand\marknote[1]{\marginnote{#1}\marker}
\newcommand\marginmarker[1][2pt]{\marginnote{\rawmarker[#1]}}

\newcommand\wordunsure[1]{\textcolor{red}{#1\marginmarker}\xspace}

% should be translated later
\newcommand\tr[1]{\textcolor{orange}{#1\marginmarker}\xspace}
% comma?
\newcommand\co{\textbf{\color{orange},\marginmarker[1pt]}\xspace}
\newcommand\TODO[1]{\noindent{\color{red}\bfseries\setlength{\fboxrule}{3pt}\fbox{TODO} #1}}
\newenvironment{deleted}{\begingroup\color{gray}}{\endgroup}
% annotation (structure, ..)
\newcommand\annot[1]{\textcolor{purple}{#1\marginmarker}\xspace}

% helper commands to test alignment
\newcommand\helpRuleHoriz[1][blue]{\tikz[remember picture,overlay,very thin,draw=#1] \draw
  (current page.east |-, 0)   -- (current page.west |-, 0)
  (current page.east |-, 1ex) -- (current page.west |-, 1ex) ;}
\newcommand\helpRuleVert[1][blue]{\tikz[remember picture,overlay,very thin,draw=#1] \draw
  (0, |- current page.north) -- (0, |- current page.south) ;}