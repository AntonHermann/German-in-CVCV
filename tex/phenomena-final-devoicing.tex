\subsection{Final devoicing}

Final devoicing is said to apply in Coda contexts,
where it makes obstruents voiceless.

\paragraph{Extrasyllabic consonants}
Theories deploying extrasyllabic consonants might
-- depending on their specific description --
have a problem here:

If extrasyllabic consonants are -- as their name suggests --
not part of the syllable, they can't be part of a Coda
and therefore not affected by final devoicing.
However, this is the case,
as \ex{Magd} \ti{[ma:kt]}, \ex{Jagd} \ti{[ja:kt]} show.

A potential solution would be to include extrasyllabic
consonants in the context final devoicing applies in.
While properly describing the context, this results
in a disjunctive description.

\cite{scheer2004} brings forth a number of arguments
against extrasyllabic consonants and \CVCV doesn't
have this context; as we will see shortly, those
consonants that are analyzed as extrasyllabic in other
theories are regular Coda consonants in \CVCV.

\paragraph{\CVCV captures the context of final devoicing correctly.}
A Coda can be summarized as those consonants following
a vowel that aren't part of an Onset
(in a tree-shaped syllable model) and since single Onset
consonants are governed and licensed by the following
Nucleus and branching Onsets by \gls{IG},
the definition of Codas in \CVCV matches exactly those
consonants that are in a Coda in tree-shaped models
and -- depending on the lateral actorship of \glspl{FEN} --
also those consonants otherwise defined as extrasyllabic.

The triplet \ex{jagen}, \ex{Jagd}, \ex{Jagden}
demonstrates this: Because of \ex{jagen} we know
that the |g| is underlying and \ex{Jagden} shows
that the |d| (which would be extrasyllabic in \ex{Jagd})
is respresented as \ti{/d/} in the lexicon.
Final devoicing applies to both, as \ti{[ja:kt]} shows.

\begin{structure}{\ex{jagen} \ti{[ja:.g@n]}}
    \drawCV{3}
    \wordstart
    \C{j}
    \longV{a}
    \C{g}
    \V{\textschwa}
    \C{n}
    \fen
\end{structure}

\begin{structure}{\ex{Jagden} \ti{[ja:k.d@n]}}
    \drawCV{4}
    \wordstart
    \C{j}
    \longV{a}
    \C{k}
    \emptyV[gov]
    \C{d}
    \V[floating]{\textschwa}
    \C{n}
    \fen
\end{structure}

\begin{structure}{\ex{Jagd} \ti{[ja:kt]}}
  \drawCV{3}
  \wordstart
  \C{j}
  \longV{a}
  \C{k}
  \emptyV[gov]
  \C{t}
  \fen
\end{structure}

\paragraph{\CVCV predicts this.}
Instead of merely describing in which context
final devoicing applies, \CVCV goes a step further
and predicts that processes such as final devoicing
are likely to occur in such a context.

\Cref{tab:lenition processes}, taken from \cite{scheer2004}, lists
different lenition processes and the context they apply in usually.
Both the Coda and intervocalic consonants are predicted to be
\emph{weak}, i.e. if lenition processes occur in a language, \CVCV
predicts them to occur on those contexts.

Coda consonants are neither governed nor licensed
whereas intervocalic consonants are both governed and licensed\footnotemark,
so their respective effects \enquote{cancel out}.
\footnotetext{In \cite{scheer2012}, the definitions of Government and
Licensing are changed slightly: only one force can affect any single segment,
if both would be eligible, Government \enquote{wins}. In consequence
intervocalic consonants would be governed only and should
therefore be even weaker than Coda consonants, altough the only empirical
hint for this is the occurence of spirantisation patterns. More evidence
is needed to support or contradict this prediction. Either way, this
distinction is not relevant for this essay.}
\TODO{incorporate predictions for the different positions}

\begin{table}
  \begin{tabular}{l l l}
    \parbox[b]{16em}{\raggedright process that affect a segment because of
    its position in the string}          & Coda    & \ctx{V\_\_V} \\
  \midrule
    devoicing                            & typical & highly improbable \\
    deaspiration (C\ti{\super h}\to C)   & typical & highly improbable \\
    velarisation (l,n\to\ti{\l,N})       & typical & highly improbable \\
    s-debuccalisation (s\to h)           & typical & highly improbable \\
    liquid gliding (r,l\to j)            & typical & highly improbable \\
    depalatalisation (\textltailn\to n)  & typical & highly improbable \\
    l-vocalisation (\ti{\l}\to w/o)      & typical & highly improbable \\
    r-vocalisation/ loss ([kaad] \enquote{card}) & typical & highly improbable \\
    {[}NC{]}\textsubscript{hom}: homorganisation of nasals  & typical & highly improbable \\
    spirantisation (b,d,g\to\ti{B,D,G})  & only if also in \ctx{V\_V} & typical \\
    voicing (t\to d)                     & highly improbable & typical \\
    rhotacism (s,z\to r)                 & highly improbable & typical \\
  \end{tabular}
  \captionsource{Different lenition processes and the context they typically apply in}{\cite[p.~144, fig.~111]{scheer2004}}
  \label{tab:lenition processes}
\end{table}