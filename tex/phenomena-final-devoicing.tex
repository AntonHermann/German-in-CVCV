\subsection{Final devoicing}

Final devoicing is said to apply in Coda contexts,
where it makes obstruents voiceless.

\paragraph{Extrasyllabic consonants}
Theories deploying extrasyllabic consonants might
-- depending on their specific description --
have a problem here:

If extrasyllabic consonants are -- as their name \tr{says} --
not part of the syllable, they can't be part of a Coda
and therefore not affected by final devoicing.
This however \tr{(however this?)} is the case,
as \ex{Magd} \ti{[ma:kt]}, \ex{Jagd} \ti{[ja:kt]} show.

A potential solution would be to include extrasyllabic
consonants in the context final devoicing applies in.
While properly describing the context, this results
in a disjunctive description.

\cite{scheer2004} brings forth a number of arguments
against extrasyllabic consonants and \CVCV doesn't
have this context; as we will see shortly, those
consonants that are analyzed as extrasyllabic in other
theories are regular Coda consonants in \CVCV.

\paragraph{\CVCV captures the context of final devoicing correctly.}
A Coda can be summarized as those consonants following
a vowel that aren't part of an Onset
(in a tree-shaped syllable model) and since single Onset
consonants are governed and licensed by the following
Nucleus and branching Onsets by \gls{IG},
the definition of Codas in \CVCV matches exactly those
consonants that are in a Coda in tree-shaped models
and -- depending on the lateral actorship of \glspl{FEN} --
also those consonants otherwise defined as extrasyllabic.

The triplet \ex{jagen}, \ex{Jagd}, \ex{Jagden}
demonstrates this: Because of \ex{jagen} we know
that the |g| is underlying and \ex{Jagden} shows
that the |d| (which would be extrasyllabic in \ex{Jagd})
is respresented as \ti{/d/} in the lexicon.
Final devoicing applies to both, as \ti{[ja:kt]} shows.

\ti{[ja:.g@n]}
\begin{structure}
    \drawCV{4}
    \wordstart
    \C{j}
    \longV{a}
    \C{g}
    \V{\textschwa}
    \C{n}
    \fen
    \Gov{V2}{V1}
\end{structure}

\ti{[ja:k.d@n]}
\begin{structure}
    \drawCV{5}
    \wordstart
    \C{j}
    \longV{a}
    \C{k}
    \emptyV
    \C{d}
    \V[floating]{\textschwa}
    \C{n}
    \fen
    \Gov{V2}{V1}
    \Gov{V5}{V4}
\end{structure}

\ti{[ja:kt]}
\begin{structure}
  \drawCV{4}
  \wordstart
  \C{j}
  \longV{a}
  \C{k}
  \emptyV
  \C{t}
  \fen
  \Gov{V2}{V1}
  \Gov{V5}{V4}
\end{structure}

\paragraph{\CVCV predicts this.}
Instead of merely describing in which context
final devoicing applies, \CVCV goes a step further
and predicts that processes such as final devoicing
are likely to occur in such a context:
--- QUOTE ---

\vspace{10pt}\TODO{\normalfont
Context: word-final vs. internal codas;
"However, the reverse situation where a process would only apply to word-final Codas
and/or their preceding vowels, but leave their internal peers untouched, does not seem to exist."

\textquote[Footnote 340 (p.~637)]{%
  At first sight, of course, final devoicing appears to be a massive counterexample to
  this generalisation. However, the status of this phenomenon is not clear.
  For instance, should it be viewed as a lenition or a fortition?
  In case the latter option is chosen, it stands aside the rest of the Coda phenomenology.
  This possibility may actually be provided for by the theory, see §543g.
  In any event, final devoicing has always been an alien in phonology.
  Its status as an UFO is independent of any particular theory.%
}
\par\textbf{-> doesn't apply because in German it's syllable-final not word-final}
}