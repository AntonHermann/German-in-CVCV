\subsection{Phonological objects in \CVCV}
\label{subsec:intro:obj}

This section briefly introduces the representation
of different phonological objects in the language of
\CVCV.

\TODO{Reihenfolge}

\subsubsection{Long vowels}
\begin{structure}{}
  \drawCV{1}
  \C{C}
  \longV{V}
\end{structure}
\TODO{complement needs to be licensed}


\subsubsection{Diphtongs}
\begin{structure}{\emph{some} Relation}
  \drawCV{1}
  \C{C}
  \V{V\textsubscript{1}}
  \emptyC
  \V{V\textsubscript{2}}
  \draw[dashed] (pV2) -- (V1);
\end{structure}

\begin{structure}{association to 2 Nuclei}
  \drawCV{1}
  \C{C}
  \V{V\textsubscript{1}}
  \emptyC
  \V{V\textsubscript{2}}
  \draw (pV2) -- (V1);
\end{structure}


\subsubsection{Coda}\label{intro:obj:coda}
Coda consonants are defined as Onsets that occur before
a governed empty Nucleus.\cite[p.~192]{scheer2004}
\begin{structure}{internal Coda}
  \drawCV{2}
  \V{V}
  \C{C}
  \C{\bfseries C}
  \emptyV[gov]
  \C{C}
  \V{V}
\end{structure}

\begin{structure}{final Coda}
  \drawCV{1}
  \V{V}
  \C{C}
  \C{\bfseries C}
  \fen
\end{structure}


\subsubsection{(Coda-Mirror) <- unten verwendet? Sonst nur brief mention}
\subsubsection{Margin contexts:}
\subsubsection{Word-start \ctx{\#\_}}\label{intro:obj:word start}
\begin{structure}{}
  \drawCV{0}
  \wordstart
  \C{C}
  \V{V}
\end{structure}


\subsubsection{Word-end \ctx{\_\#}}\label{intro:obj:word end}
\subsubsection{Syllabic consonants}
\subsubsection{Branching Onsets (Infrasegmental Government)}
\subsubsection{Vowel-Zero-Alternations}