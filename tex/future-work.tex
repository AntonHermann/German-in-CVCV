

\subsection{IG between plosive+fricative sequences?}


\subsection{More on Issatschenko \tr{vs.} \CVCV}
Look further into the intersections of Issatschenkos rules
and \CVCV. Where they overlap, how do both theories reason about those configurations?
In which cases are their predictions different?
Maybe there are mechanisms in \CVCV that could provide explanations that I haven't considered,
both in \cite{scheer2004} and other work in this area of research.

Maybe some assumptions of the theory could be adjusted to better capture the data
by incorporating insights from this existing work.

\TODO{letzter Teil eigentlich nicht auf Issatschenko beschränkt, vllt eher separat allgemein
formulieren, bspw. unten bei Nennung von Höhle \& Vater}


\subsection{Issatschenkos Synkopierungsregel $S$ \& Verbalsuffixregel $D$}

Verbalsuffixregel $D$:

\(
  \{\schwaMobi\} \to \{\schwaCons\} \,/\,
  \phongroupi[l]{ t d \\ \schwaMobi + Nasal } \phonctx \phongroupi{st\\t}_\textsc{Verbalsuffix}
\)

\(
  \{\schwaMobi\} \to \{\schwaCons\} \,/\,
  \renewcommand{\arraystretch}{1.2}%
  \left\{
    \begin{tabular}{@{\ }l@{\ }}t d\\\schwaMobi + Nasal\end{tabular}
  \right\}
  \phonctx
  \left\{
    \begin{tabular}{c}st\\t\end{tabular}
  \right\}_\textsc{Verbalsuffix}
\)


\subsection{More schwa contexts}
\begin{itemize}
\item Adjectives (\ex{dunkel} vs. \ex{dunkle})
\item Verbs
\item Verbalpronomina: \ex{Gelache, Gequake, Geläute, Gesumme},
  wenn Verbalstamm auf Schwa+Liquid endet kein -e:
  \ex{Gejammer, Geknister, Geklapper, Gemurmel, Gerassel}
\end{itemize}


Explore mapping other theories to the \CVCV model.
Rule~18 in \cite[p.~176]{hohleVater1978} (\cref{fig:HöhleVater_Rule18})
tries to capture a complex context where syllabic consonants appear,
along with a discussion of its limitations,
where it overgenerates and how different \tr{Wortarten} induce different
patterns in the \tr{phonological} realisation.

\begin{figure}
  \begin{enumerate}[label=(\roman*)]
  \item  \(\displaystyle
    \feat{--\,obstr} \tikzmarknode{arrow}{\rightarrow} \feat{+silb} /\,\#\,(
      \,\text{X}\;
      \phongroup{
        \feat{\tikzmarknode{a}{+}\,obstr}\hspace*{8.5pt}\\
        \feat{\tikzmarknode{b}{--}\,obstr\\+\,nasal}\hspace*{8.5pt}\\[8pt]
        \phongroup{
          \feat{--\,obstr\\\tikzmarknode{c}{--}\,nasal}\\
          \feat{\tikzmarknode{d}{--}\,kons}
        }+
      }
    )
    \,\phonctx\,
    (\feat{--\,kons}\,\text{Y})
    \ \#
    \tikz[overlay,remember picture]{
      \foreach \x in {a,b,c,d}{
        \draw (\x.west -| arrow) node {(\x)};
      }
    }
  \)

  \item Bedingung:
    \begin{tabular}[t]{ll}
      (a)      & obligatorisch \\
      (b), (c) & fakultativ für Nasale,\\
              & obligatorisch für Liquide \\
      (d)      & fakultativ
    \end{tabular}
  \end{enumerate}
  \caption{Rule~18 of \cite{hohleVater1978} tries to capture the contexts
  where syllabic sonorants occur.}
  \label{fig:HöhleVater_Rule18}
\end{figure}

As with Issatschenkos theory, a variety of different aspects
can be explored by comparing the \CVCV model with
\tr{the results of/the theory of/$\emptyset$}
\cite{hohleVater1978}:
\begin{itemize}
  \item the interaction between vowel-zero alternation and syllabic consonants,
  \item the influcence of \tr{Wortarten} on phonological regularities.
  \item how this derivational analysis can be expressed in a theory
    that claims to \tr{ablehnen} derivations.\footnotemark
\end{itemize}
\footnotetext{The foreword of \cite{scheer2012} provides a slightly different view here
  than \gls{GP} simply \tr{ablehnen} derivations:
  \enquote{This book is about the way morpho-syntax talks to phonology
    % (this direction, the other is only touched on in passing, see §127).
    [\textellipsis].
    Or rather, it is only about one of the two means that allow morpho-syntax to bear on phonology:
    the representational side of the coin (as opposed to its procedural side, i.e. cyclic derivation, today phase theory,
    % see §§5f).%
    [\textellipsis]).%
  }
}

Wortarten: Influence Syntax/Morphologie auf Phon
  \to what mechanisms \tr{does} the Direct Interface \cite{scheer2012}

\subsection{}
Unterschiedliche Transparenz von \textschwa{} für Umlaut
beim Diminutiv / Unterschiede -er, -e und -en:
\par \ex{Vater} \to{} \ex{Väterchen}
\par \ex{Haken} \to{} \ex{Häkchen}