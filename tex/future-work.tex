

\subsection{IG between plosive+fricative sequences?}


\subsection{More on Issatschenko \tr{vs.} \CVCV}
Look further into the intersections of Issatschenkos rules
and \CVCV. Where they overlap, how do both theories reason about those configurations?
In which cases are their predictions different?
Maybe there are mechanisms in \CVCV that could provide explanations that I haven't considered,
both in \cite{scheer2004} and other work in this area of research.

Maybe some assumptions of the theory could be adjusted to better capture the data
by incorporating insights from this existing work.

\TODO{letzter Teil eigentlich nicht auf Issatschenko beschränkt, vllt eher separat allgemein
formulieren, bspw. unten bei Nennung von Höhle \& Vater}


\subsection{Issatschenkos Synkopierungsregel $S$}


\subsection{More schwa contexts}
\begin{itemize}
\item Adjectives (\ex{dunkel} vs. \ex{dunkle})
\item Verbs
\item Verbalpronomina: \ex{Gelache, Gequake, Geläute, Gesumme},
  wenn Verbalstamm auf Schwa+Liquid endet kein -e:
  \ex{Gejammer, Geknister, Geklapper, Gemurmel, Gerassel}
\end{itemize}


Explore mapping other theories to the \CVCV model.
Rule~18 in \cite[p.~176]{hohleVater1978} tries to capture a complex context where
syllabic consonants appear, along with a discussion of situations where this
rule doesn't suffice\marknote{überprüfen}.

\newcommand\feat[1]{%
  \ensuremath{\left[%
    \renewcommand{\arraystretch}{1}%
    \begin{tabular}{@{}c@{}}#1\end{tabular}%
  \right]}%
}

\newcommand\phongroup[1]{%
  \scalebox{0.5}[1]{\ensuremath{\displaystyle%
    \left\{\scalebox{2}[1]{%
      \renewcommand{\arraystretch}{1.5}%
      \begin{tabular}{@{}c@{}}#1\end{tabular}%
    }\right\}%
  }}%
}

\newcommand\phonctx{\rule[-1.5pt]{4ex}{.5pt}}

% \newlength{\mylength}
% \settowidth{\mylength}{+}
% \showthe\mylength

\[
  \feat{--\,obstr} \rightarrow \feat{+silb} /\,\#\,(
    \,\text{X}\;
    \phongroup{
      \feat{+\,obstr}\hspace*{8.5pt}\\
      \feat{--\,obstr\\+\,nasal}\hspace*{8.5pt}\\[8pt]
      \phongroup{
        \feat{--\,obstr\\--\,nasal}\\
        \feat{--\,kons}
      }+
    }
  )
  \,\phonctx\,
  (\feat{--\,kons}\,\text{Y})
  \ \#
\]