\subsection{Government and Licensing}
\label{subsec:intro:gov lic}

In \CVCV, Government and Licensing are opposing forces:
While Government inhibits the melodic expression of
its target, Licensing makes it stronger.

Where in \textcite{scheer2004} both forces are independent
from each other (a constituent can be both governed
and licensed, in which case both forces cancel out),
\textcite{scheer2012} refines the model as follows:
\blockquote[\cite{scheer2012}]{
  (68) government over licensing
  
  no constituent can be governed and licensed at the
  same time. In case a constituent can potentially be
  subject to both lateral forces, it will be governed.
}
\TODO{is this relevant to German? maybe just mention but ignore it afterwards}

Each Nucleus always \wordunsure{exhibits}
all available forces: full vowels govern and license
while the ability of schwa and final empty Nuclei to
govern/license is subject to (separate) language-specific
parametrization.
In german, both schwa and final empty Nuclei govern but
do not license.

\textquote[{\cite[p.~144]{scheer2012}}]{
  Government and licensing must be counted out because they represent
  phonological computation, rather than representational objects,
  and hence cannot be stored in the lexicon (see §186).
}

\subsection[TODO]{\TODO{}}
\begin{itemize}\color{red}
  \item Havlík: every other alternation site is vocalised,
    counting from the right edge of the sequence\par
    \q[p.~508]{scheer2004}{They may be informally stated as under (306) below.
    \begin{itemize}[widest=(306), leftmargin=*]
    \item[(306)] the two patterns of vowel-zero alternations
      \begin{enumerate}[label=\alph*.]
        \item Havlík\\
          given a chain of alternation sites, vocalise every other one,
          counting from the right margin.
        \item Lower\\
          given a chain of alternation sites, vocalise all of them
          save the last one.
      \end{enumerate}
    \end{itemize}}
  \item \gls{ECP}
\end{itemize}