\subsection{Government and Licensing}
\label{subsec:intro:gov lic}

\CVCV \tr{übernimmt} the lateral relations
\emph{Government} and \emph{Licensing} from
\SGP, it however changes their definitions:

\TODO{Werden Gov+Lic in SGP auch für positional strength /
zur Erklärung von Lenition/Fortition Prozesse verwendet?

\small Wenn nicht hat sich auch die Funktion der Relationen erweitert,
von Erklärungen von Strukturbegrenzungen zusätzlich auch
zur Erklärung von Kontexten von "weicheren" phon. Prozessen
als "harter" Beschränkung: Assimilationen, Alternationen,
Merkmalsänderungen. (vllt ist dieser Unterschied aber gar nicht
so erwähnenswert)}

In \CVCV, Government and Licensing are opposing forces:
While Government inhibits the melodic expression of
its target, Licensing makes it stronger.

Each Nucleus always \wordunsure{exhibits}
all available forces\footnotemark : full vowels govern and license
while the ability of schwa and final empty Nuclei to
govern/license is subject to (separate) language-specific
parametrization.
In german, both schwa and final empty Nuclei govern but
do not license (see \cref{subsec:params:schwa} \& \cref{subsec:params:fen}).
\footnotetext{%
Where in \textcite{scheer2004} both forces are independent
from each other (a constituent can be both governed
and licensed, in which case both forces cancel out),
\textcite{scheer2012} refines the model as follows:
\blockquote{%[\cite{scheer2012}]
  (68) government over licensing
  
  no constituent can be governed and licensed at the
  same time. In case a constituent can potentially be
  subject to both lateral forces, it will be governed.
}

This change is made to rule out the so-called \emph{nightmare}
context where a consonant is governed but not licensed
(because \acrshort{FEN} are parametrized to license but not govern
or because the following vowel is called to license the
complement of a left-headed long vowel in languages that show
vowel length alternations). The original definition predicts
this context to be even weaker than Codas or intervocalic
consonants, a situation that seems to lack any evidence.

Because both theoretically possible occurences of the nightmare
context are impossible with the parametrization of German,
the following text uses the original definition.
All analyses should however work equally with the updated definitions
of government, licensing and the resulting changes to the
descriptions of Coda, Coda Mirror, etc.}

\textquote[{\cite[p.~144]{scheer2012}}]{
  Government and licensing must be counted out because they represent
  phonological computation, rather than representational objects,
  and hence cannot be stored in the lexicon (see §186).
}