\subsection{Empty Nuclei}
\label{subsec:intro:empty N}

Another trademark of \SGP is the deployment of
empty Nuclei:
\TODO{where are they used? primarily word-final?\\
  \implies{} word-final, vowel-zero alternation, Coda-Onset Zusammenhang
}

\CVCV takes the use of empty Elements to the extreme:
It assumes a strict sequence of alternating Onsets and Nuclei.
\tr{Whereever} an uninterrupted sequence of vowels or consonants
surfaces, the theory assumes the presence of empty segments
\tr{inbetween/between them}.

What is commonly identified as a branching Onset is analyzed
as a sequence of unary Onsets followed by empty Nuclei,
between the consonants of a Coda
(whose formal identity in \CVCV will be introduced in \cref{intro:obj:coda})
is an empty Nucleus and a series of vowels \tr{encloses/enclose} empty Onsets.

Every word consists entirely of Onset-Nucleus pairs and
if the word is vowel-final on the surface,
the underlying structure contains an empty Nucleus.
In some languages, there even is a fully empty Onset-Coda pair before a word,
marking the left edge. These margin markers are introduced in \cref{intro:obj:word start}
and \cref{intro:obj:word end}, respectively.

\TODO{empty category principle?}

\textquote[{\cite[p.~67]{scheer2004}}]{%
  \begin{definition}[53]\textbf{Empty Category Principle -- final version}
    
    a Nucleus may remain phonetically unexpressed iff it is
    \begin{enumerate}
      \item properly governed or
      \item enclosed within a domain of Infrasegmental Government or
      \item domain-final
    \end{enumerate}
  \end{definition}%
}

\TODO{Überleitung á la \enquote{das kommt jetzt:}}