\subsection{Government \& Licensing in (Standard) Government Phonology}
\label{subsec:intro:gov lic in SGP}

Government and licensing are binary relations between two
skeletal positions. These relations are asymmetric, i.e.
one of the positions acts as the \emph{head} of the relation and
is called \emph{governor}/\emph{licensor} while its counterpart
is called \emph{governee}/\emph{licensee}.

In \Cite{kaye1990}, every position in a domain
(this notion will be properly introduced in \cref{subsec:intro:domains})
except one has to be licensed, the one unlicensed position is the
head of the domain. Government is a special form of licensing with
additional conditions: segments have to be adjacent on the skeletal layer
(or \emph{projection} in \gls{GP} terms) and the relation has a fixed
direction, which is a universal property independent of a specific language.

The distinction between government and licensing is not clear;
neither is whether the adjacency and directionality conditions apply
to both relations: they explicitly apply to government, but one consequence
of these conditions, that the relation is necessarily binary\footnotemark,
seems to be implicitly assumed for licensing as well.
\footnotetext{A ternary relation necessarily violates one of the conditions:
Either the head of the relation is located inbetween the other positions, violating
fixed directionality, or one of the position is not adjacent to the head.}

In \gls{GP} there are a number of different kinds of government/licensing,
one of which is constituent government which is used to define the syllabic
constituents Onset, Nucleus and Rhyme. 

\q[p.~149]{scheer2004}{Nuclei govern their Onsets, Onsets govern preceding Codas, the first members of branching Onsets and branching Nuclei govern the dependent part of the constituent.}