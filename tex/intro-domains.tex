\subsection{Domains}



\subsubsection{Suffixes}

\begin{itemize}\color{red}
  \item analytic vs non-analytic suffixes\par
  \q[p.~630]{scheer2004}{If in addition the suffix \ex{-gar} is analytic
    (see §§24,249,349,630 for discussion of this concept),
    i.e. if it does not enter the domain of the root,
    the underlying structure of the word is \ti{/((sa\l\o)-gar)/}.}\par
  \q[p.~631]{scheer2004}{On the other hand, all vowel-initial suffixes
    are necessarily nonanalytic under this analysis. That is,
    their vowel sits in the root-final empty Nucleus. Analytic
    vowel-initial suffixes would create the structure
    \ti{[(\textellipsis\l\o)-\*{V}]}. If it is true that l-vocalisation takes
    place domain-finally, then, the \ti{[\l]} should vocalise in
    intervocalic position, which of course it does not.}
\end{itemize}