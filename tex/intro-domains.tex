\subsection{Domains}

Phonological domains are the \gls{GP}-equivalent of
\emph{(transformational) cycles} introduced in SPE \parencite{chomsky1968}
or \emph{phases} of the more recent \emph{phase theory}.

They represent sequences that are processed as a whole by phonology, i.e.
\q[p.~22]{scheer2004}{without interference from other modules of the grammar.}.
\Cite{scheer2012} discusses the specific nature of domains in greater
detail, including an argument against brackets and in favor of
interactionism, but for this essay\co the only important aspects of
domains are the special properties of domain-final empty Nuclei and
their influence on suffixes.


\subsubsection{Final empty Nuclei}
Since the basic structure of the theory is a strict CV sequence,
no domain ends in a consonant; those consonants that appear as final
in the phonetic output are actually Onsets of \gls{FEN}
-- a trademark of \gls{GP}.

How final Nuclei behave is dictated by a number of parameters.
In most languages -- those that allow words to end with a consonant --
\gls{FEN} are governed, i.e. they are allowed to be empty even without
another vowel dispensing Government.
In languages like Italian final Nuclei are not governed and therefore
have to be contentful.

Unlike other empty Nuclei and depending on the language specific parameters,
\gls{FEN} can be laterally enabled to govern, license\co or both.
Their licensing ability determines whether word-final consonants behave
like Coda consonants (if so, they don't license).
If they are allowed to govern, word-final consonant clusters as in \ex{Haupt}
are possible. \parencite[appendix~1]{scheer2004}

Their governing ability is however restricted in one case:
\q[p.~643]{scheer2004}{Final empty Nuclei can only govern Nuclei that are
  bare of any underlying melody (floating or attached)}.
In consequence, no alternation site surfaces as zero in the context
\ctx{\_C\#}.


\subsubsection{Analytic vs. non-analytic suffixes}
There are two classes of suffixes called \enquote{level 1} and
\enquote{level 2} suffixes in Lexical Phonology and
\enquote{analytic} / \enquote{non-analytic} by \tr{Kaye (1995)},
one of the \tr{founders} of \gls{GP}.

Analytic suffixes form a separate domain, i.e. both the root and the
suffix are processed independently\co while non-analytic suffixes
enter the domain of the root. The combined sequence is then processed
as a single unit without a boundary visible to phonology between them.


\subsubsection{Vowel-initial suffixes}
\Cite{scheer2004} argues that vowel-initial suffixes are non-analytic
based on the derivation of Polish \ex{pieseczek} (p.~584) and
l-vocalisation in Brazilian Portuguese (p.~631).
It isn't clear whether this analysis is meant to be a universal
property of vowel-initial suffixes or if analytic properties are
possible in other languages and if so, how their structure is captured
by \CVCV\footnotemark.
\footnotetext{\samepage Earlier \gls{GP} work by Gussmann \& Kaye analysed
\ex{pieseczek}, consisting of the three morphemes \ex{pies-ecz-ek}, under
the assumption of analytic suffixes, resulting in the structure
\ti{(((p\o s\o)\o cz\o)\o k\o)} (\ti{\o} = empty Nucleus). They then
introduce a reduction rule that eliminates empty Nuclei with empty Onsets
(the \ti{\o\o} sequences around the domain boundaries). This analysis is
however rejected by \citeauthor{scheer2004} based on the extra rule needed,
it's dependence on cyclicity (which \CVCV tries to get rid of) and the
arbitrary assignment of different Polish suffixes to the categories
\enquote{anayltic}/\enquote{non-analytic}.}

In \citeauthor{scheer2004}s analysis vowel-initial suffixes join the domain
of the root\marknote{correct term?} they are attached to,
their initial vowel then takes the position of the \gls{FEN}.
What happens when the root ends with a vowel and the final Nucleus therefore has
content is not specified, however the only phenomenon noted where a full vowel is
influenced by lateral relations is that of vowel-length alternations, which don't
occur in German either way, so this detail bears no relevance for this essay.

% \begin{itemize}\color{red}\footnotesize
%   \item analytic vs non-analytic suffixes\par
%   \q[p.~630]{scheer2004}{If in addition the suffix \ex{-gar} is analytic
%     (see §§24,249,349,630 for discussion of this concept),
%     i.e. if it does not enter the domain of the root,
%     the underlying structure of the word is \ti{/((sa\l\o)-gar)/}.}\par
%   \q[p.~631]{scheer2004}{On the other hand, all vowel-initial suffixes
%     are necessarily nonanalytic under this analysis. That is,
%     their vowel sits in the root-final empty Nucleus. Analytic
%     vowel-initial suffixes would create the structure
%     \ti{[(\textellipsis\l\o)-\*{V}]}. If it is true that l-vocalisation takes
%     place domain-finally, then, the \ti{[\l]} should vocalise in
%     intervocalic position, which of course it does not.}
%   \item analytic = level 1, non-analytic = level 2
%     \cite[p.~295, note~166]{scheer2004}
% \end{itemize}