\subsection{Schwa distribution}

\subsubsection{Alternating vowels in \CVCV}
Because of the multiple vowels that alternate with zero found in Slavik
languages, \cite{scheer2004} abandons the insertion analysis of earlier
\SGP in favor of a deletion analysis.

Another argument brought forward is that where \SGP only differentiates
between final and alternating empty Nuclei, in \CVCV there are
\q[p.~90]{scheer2004}{four kinds of \enquote{empty} Nuclei:
1) \ctx{T\_R}, 2) \ctx{R\_T}, 3) \ctx{C\_\#}, 4) alternation sites}.
Because only 4) shows vowel-zero alternations and the other types
remain empty in all circumstances, he argues that not all empty
Nuclei could be possible alternation sites:
\q[p.~90]{scheer2004}{If it were true that all empty Nuclei
are alternation sites which are filled in by default, there should be
no empty Nuclei that never appear on the surface.}%
\footnote{This argument doesn't seem logical to me.
  As we will see shortly, alternating vowels only manifest phonetically
  if they fail to be governed. 1) to 3) all have in common
  that their empty Nucleus has to be governed anyway, so it isn't
  clear why those can't be theoretical alternation sites where
  the vowel is never inserted because the Nucleus is always governed.}

Alternating vowels are represented as \enquote{floating melodies} in \CVCV:
Both the melody and the structure (V-slots) are underlyingly present
but alternating vowels are not attached to their Nucleus.
If this Nucleus is governed, the weakening effect of Government
prevents the association of the melody and doesn't surface, otherwise
the melody attaches to the Nucleus and the vowel surfaces.

\Cref{fig:3 kinds of vowels} shows the three kinds of vowels
\CVCV differentiates: full vowels, alternating vowels and empty Nuclei.

\TODO{only 2 / 3 kinds of empty Nuclei shown.
Either mention this or show only 1 example for empty Nuclei}

\begin{figure}[h]
  \newcommand\structureSpacing{1.4}
  \centering
  \begin{tabular}{ p{81pt} p{81pt} p{104pt} p{104pt} }
    \multicolumn{2}{p{162pt}}{\centering Nuclei that are lexically endowed with melody} &
    \multicolumn{2}{p{200pt}}{\centering Nuclei that lack any lexical specification for melody} \\

    a. the melody is associated: &
    \raggedright b. the melody floates:  &
    \multicolumn{2}{p{200pt}}{\centering
      c. empty Nuclei never appear\newline on the surface} \\

    full vowels &
    vowels that alternate with zero &
    branching Onset \newline \ctx{VT\ti{\o}RV} &
    Coda-Onset      \newline \ctx{VR\ti{\o}TV} \\

    \begin{structure}[\structureSpacing]
      \drawCV{1}
      \C{C}
      \V{\ti{E}}
      \C{C}
    \end{structure} &
    \begin{structure}[\structureSpacing]
      \drawCV{1}
      \C{C}
      \V[floating]{\ti{E}}
      \C{C}
    \end{structure} &
    \begin{structure}[\structureSpacing]
      \drawVCV{2}
      \V{V}
      \C{T}
      \emptyV
      \C{R}
      \V{V}
      \IG{C2}{C1}
      \Lic{V2}{C2}
    \end{structure} &
    \begin{structure}[\structureSpacing]
      \drawVCV{2}
      \V{V}
      \C{R}
      \emptyV
      \C{T}
      \V{V}
      \Gov{V2}{V1}
    \end{structure} \\
    \centering cz \ti{/lEs/} \enquote{forest} &
    \centering cz \ti{/pEs/} \enquote{dog} &
    \centering fr \ti{/patri/} patrie \newline\enquote{fatherland} &
    \centering fr \ti{/parti/} parti  \newline\enquote{part}
  \end{tabular}
  \captionsource{CVCV: three different types of Nuclei}{\cite[p.~92, fig.~73]{scheer2012}}
  \label{fig:3 kinds of vowels}
\end{figure}

{\color{purple}DRAFT

\Cref{fig:3 kinds of vowels} shows: in 1 language a vowel can exist
as a full, associated vowel (\ti{/lEs/}) and
as an alternating, floating vowel (\ti{/pEs/}).

This maps precisely to \citeauthor{issatschenko1974}s
\emph{Schwa constans} and \emph{Schwa mobile}.

\TODO{Issatschenko lesen}

}