\subsection{Schwa distribution}

\subsubsection{Alternating vowels in \CVCV}
Because of the multiple vowels that alternate with zero found in Slavik
languages, \cite{scheer2004} abandons the insertion analysis of earlier
\SGP in favor of a deletion analysis.

Another argument brought forward is that where \SGP only differentiates
between final and alternating empty Nuclei, in \CVCV there are
\q[p.~90]{scheer2004}{four kinds of \enquote{empty} Nuclei:
1) \ctx{T\_R}, 2) \ctx{R\_T}, 3) \ctx{C\_\#}, 4) alternation sites}.
Because only 4) shows vowel-zero alternations and the other types
remain empty in all circumstances, he argues that not all empty
Nuclei could be possible alternation sites:
\q[p.~90]{scheer2004}{If it were true that all empty Nuclei
are alternation sites which are filled in by default, there should be
no empty Nuclei that never appear on the surface.}%
\footnote{This argument doesn't seem logical to me.
  As we will see shortly, alternating vowels only manifest phonetically
  if they fail to be governed. 1) to 3) all have in common
  that their empty Nucleus has to be governed anyway, so it isn't
  clear why those can't be theoretical alternation sites where
  the vowel is never inserted because the Nucleus is always governed.}

Alternating vowels are represented as \enquote{floating melodies} in \CVCV:
Both the melody and the structure (V-slots) are underlyingly present
but alternating vowels are not attached to their Nucleus.
If this Nucleus is governed, the weakening effect of Government
prevents the association of the melody and doesn't surface, otherwise
the melody attaches to the Nucleus and the vowel surfaces.

\Cref{fig:3 kinds of vowels} shows the three kinds of vowels
\CVCV differentiates: full vowels, alternating vowels and empty Nuclei.

\TODO{only 2 / 3 kinds of empty Nuclei shown.
Either mention this or show only 1 example for empty Nuclei}

\begin{figure}[h]
  \newcommand\structureSpacing{1.4}
  \centering
  \begin{tabular}{ p{81pt} p{81pt} p{104pt} p{104pt} }
    \multicolumn{2}{p{162pt}}{\centering Nuclei that are lexically endowed with melody} &
    \multicolumn{2}{p{200pt}}{\centering Nuclei that lack any lexical specification for melody} \\

    a. the melody is associated: &
    \raggedright b. the melody floates:  &
    \multicolumn{2}{p{200pt}}{\centering
      c. empty Nuclei never appear\newline on the surface} \\

    full vowels &
    vowels that alternate with zero &
    branching Onset \newline \ctx{VT\ti{\o}RV} &
    Coda-Onset      \newline \ctx{VR\ti{\o}TV} \\

    \begin{structure}[\structureSpacing]
      \drawCV{1}
      \C{C}
      \V{\ti{E}}
      \C{C}
    \end{structure} &
    \begin{structure}[\structureSpacing]
      \drawCV{1}
      \C{C}
      \V[floating]{\ti{E}}
      \C{C}
    \end{structure} &
    \begin{structure}[\structureSpacing]
      \drawVCV{2}
      \V{V}
      \C{T}
      \emptyV
      \C{R}
      \V{V}
      \IG{C2}{C1}
      \Lic{V2}{C2}
    \end{structure} &
    \begin{structure}[\structureSpacing]
      \drawVCV{2}
      \V{V}
      \C{R}
      \emptyV
      \C{T}
      \V{V}
      \Gov{V2}{V1}
    \end{structure} \\
    \centering cz \ti{/lEs/} \enquote{forest} &
    \centering cz \ti{/pEs/} \enquote{dog} &
    \centering fr \ti{/patri/} patrie \newline\enquote{fatherland} &
    \centering fr \ti{/parti/} parti  \newline\enquote{part}
  \end{tabular}
  \captionsource{CVCV: three different types of Nuclei}{\cite[p.~92, fig.~73]{scheer2012}}
  \label{fig:3 kinds of vowels}
\end{figure}

The Czech examples in \Cref{fig:3 kinds of vowels} show that it is possible
to have a single vowel both as full (\ti{/lEs/}) and alternating (\ti{/pEs/})
vowel in the same language.

This maps precisely to \emph{Schwa constans} and \emph{Schwa mobile}
\cite{issatschenko1974}.

\subsubsection{\CVCV analysis of \citeauthor{issatschenko1974}s Schwa rules}
Based on this insight, we can now analyze whether \CVCV is able to capture
the distribution of Schwa (mobile) in german.

\citeauthor{issatschenko1974} describes 4 vocalisation rules where
a Schwa mobile turns into a Schwa constans and therefore receives
phonetic manifestation:

\begin{enumerate}
  \item \deriv
    {\ti{/\textsce/}} % E (schwa mobile)
    {\ti{/@/}}        % ə (schwa constans / realization)
    {$\_\_\begin{Bmatrix}L \\ N\end{Bmatrix}(C) \;\big]_\omega$}
  \item same as 1. but with composition boundary instead of word boundary
  \item \deriv
    {\ti{/\textsce/}} % E (schwa mobile)
    {\ti{/@/}}        % ə (schwa constans / realization)
    {$\_\_\begin{Bmatrix}L \\ N\end{Bmatrix}(C) + CX$}
  \item \deriv
    {\ti{/\textsce/}} % E (schwa mobile)
    {\ti{/@/}}        % ə (schwa constans / realization)
    {$\_\_\begin{Bmatrix}L \\ N\end{Bmatrix}(C) + \text{\textsce}\ C(C)$}
\end{enumerate}

If we translate the context of these rules into the
vocabulary of \CVCV, we can examine whether its prediction
\tr{of/on} the schwa-vocalisation matches the patterns
described by \citeauthor{issatschenko1974}.

\Cref{fig:issatschenko voc rule 1}
depicts the situation of rule~1.
While in \ref{fig:issatschenko voc rule 1:L C end}
the liquid or nasal and the consonant enclose an
internal empty Nucleus without lateral actorship,
the vowel following the Schwa in
\ref{fig:issatschenko voc rule 1:L end}
is a \gls{FEN}.

\begin{figure}[h]
  \centering
  \begin{subfigure}{.49\textwidth}
    \centering
    \begin{structure}
      \drawVCV{2}
      \V[floating]{\textschwa}
      \C{L/N}
      \emptyV
      \C{C}
      \fen
      \Gov{V2}{V1}
      \NoGov{V1}{V0}
    \end{structure}
    \caption{with extra C}
    \label{fig:issatschenko voc rule 1:L C end}
  \end{subfigure}
  \hfill
  \begin{subfigure}{.49\textwidth}
    \centering
    \begin{structure}
      \drawVCV{1}
      \V[floating]{\textschwa}
      \C{L/N}
      \fen
      \NoGov{V1}{V0}
    \end{structure}
    \caption{without extra C}
    \label{fig:issatschenko voc rule 1:L end}
  \end{subfigure}
  \caption{Issatschenko vocalisation rule 1 in \CVCV}
  \label{fig:issatschenko voc rule 1}
\end{figure}

\Glspl{FEN} in German principally have the ability
to govern, however the government relationships of
\glspl{FEN} in general \tr{cannot/can not} target
non-empty Nuclei:
\marknote{Move quote to Intro:Gov+Lic?}
\q[p.~643]{scheer2004}{Final empty Nuclei can only
govern Nuclei that are bare of any underlying melody
(floating or attached)} (see \cref{subsec:intro:gov lic}).

Albeit for different reasons, in both variants
the lack of government of $V_1$ results in the
attachment of the floating melody and the Schwa
surfaces; the prediction of \CVCV is consistend with
\citeauthor{issatschenko1974}s rule.

\TODO{Für rule 2., 3., 4.: ist domain final = FEN?}

\TODO{analyze Tilgungsregel}

\TODO{other restrictions/predictions in issatschenko?}
  Adjectives, Verbs?