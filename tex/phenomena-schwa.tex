\subsection{Schwa distribution}

\subsubsection{Alternating vowels in \CVCV}
Because of the multiple vowels that alternate with zero found in Slavik
languages, \textcite{scheer2004} abandons the insertion analysis of earlier
\SGP in favor of a deletion analysis.

Another argument brought forward is that where \SGP only differentiates
between final and alternating empty Nuclei, in \CVCV there are
\q[p.~90]{scheer2004}{four kinds of \enquote{empty} Nuclei:
1) \ctx{T\_R}, 2) \ctx{R\_T}, 3) \ctx{C\_\#}, 4) alternation sites}.
Because only 4) shows vowel-zero alternations and the other types
remain empty in all circumstances, he argues that not all empty
Nuclei could be possible alternation sites:
\q[p.~90]{scheer2004}{If it were true that all empty Nuclei
are alternation sites which are filled in by default, there should be
no empty Nuclei that never appear on the surface.}%
\footnote{This argument doesn't seem logical to me.
  As we will see shortly, alternating vowels only manifest phonetically
  if they fail to be governed. 1) to 3) all have in common
  that their empty Nucleus has to be governed anyway, so it isn't
  clear why those can't be theoretical alternation sites where
  the vowel is never inserted because the Nucleus is always governed.}

Alternating vowels are represented as \enquote{floating melodies} in \CVCV:
Both the melody and the structure (V-slots) are underlyingly present
but alternating vowels are not attached to their Nucleus.
If this Nucleus is governed, the weakening effect of Government
prevents the association of the melody and doesn't surface, otherwise
the melody attaches to the Nucleus and the vowel surfaces.

\Cref{fig:3 kinds of vowels} shows the three kinds of vowels
\CVCV differentiates: full vowels, alternating vowels and empty Nuclei.

\TODO{only 2 / 3 kinds of empty Nuclei shown.
Either mention this or show only 1 example for empty Nuclei}

\begin{figure}[h]
  \newcommand\structureSpacing{1.4}
  \centering
  \begin{tabular}{ p{81pt} p{81pt} p{104pt} p{104pt} }
    \multicolumn{2}{p{162pt}}{\centering Nuclei that are lexically endowed with melody} &
    \multicolumn{2}{p{200pt}}{\centering Nuclei that lack any lexical specification for melody} \\

    a. the melody is associated: &
    \raggedright b. the melody floates:  &
    \multicolumn{2}{p{200pt}}{\centering
      c. empty Nuclei never appear\newline on the surface} \\

    full vowels &
    vowels that alternate with zero &
    branching Onset \newline \ctx{VT\ti{\o}RV} &
    Coda-Onset      \newline \ctx{VR\ti{\o}TV} \\

    \begin{structure}[\structureSpacing]{}
      \drawCV{1}
      \C{C}
      \V{\ti{E}}
      \C{C}
    \end{structure} &
    \begin{structure}[\structureSpacing]{}
      \drawCV{1}
      \C{C}
      \V[floating]{\ti{E}}
      \C{C}
    \end{structure} &
    \begin{structure}[\structureSpacing]{}
      \drawVCV{2}
      \V{V}
      \C{T}
      \emptyV
      \C{R}
      \V{V}
      \IG{C3}{C2}
      \Lic{V3}{C3}
    \end{structure} &
    \begin{structure}[\structureSpacing]{}
      \drawVCV{2}
      \V{V}
      \C{R}
      \emptyV[gov]
      \C{T}
      \V{V}
    \end{structure} \\
    \centering cz \ti{/lEs/} \enquote{forest} &
    \centering cz \ti{/pEs/} \enquote{dog} &
    \centering fr \ti{/patri/} patrie \newline\enquote{fatherland} &
    \centering fr \ti{/parti/} parti  \newline\enquote{part}
  \end{tabular}
  \captionsource{CVCV: three different types of Nuclei}{\cite[p.~92, fig.~73]{scheer2012}}
  \label{fig:3 kinds of vowels}
\end{figure}

The Czech examples in \Cref{fig:3 kinds of vowels} show that it is possible
to have a single vowel both as full (\ti{/lEs/}) and alternating (\ti{/pEs/})
vowel in the same language.

This maps precisely to \emph{Schwa constans} and \emph{Schwa mobile} in
\textcite{issatschenko1974}.

\subsubsection{\CVCV analysis of Issatschenkos Schwa rules}
Based on this insight, we can now analyze whether \CVCV is able to capture
the distribution of Schwa (mobile) in german.

\TODO{Terminology: E and \textschwa{} for Schwa mobile and Schwa constans?
Issatschenko uses \# and E here. \# is used for boundaries nowadays but
on the other hand, as one of the goals of \CVCV is to reduce this usage
of diacritics, the \# is used way less and maybe in the context of rules,
it would be okay? Or introduce another, totally different symbol?}

\TODO{Issatschenko uses \{ \} for "morphonologische Transkriptionen".
Currently, I translated those to / / pairs, but that may be wrong?}

\citeauthor{issatschenko1974} describes 4 vocalisation rules where
a Schwa mobile turns into a Schwa constans and therefore receives
phonetic manifestation:

\begin{enumerate}
  \item \deriv
    {/\schwaMobi/}    % E (schwa mobile)
    {/\schwaCons/}    % ə (schwa constans / realization)
    {$\_\_$\{L,N\} (C) $\;\big]_\omega$}
  \item same as 1. but with composition boundary instead of word boundary
  \item \deriv
    {/\schwaMobi/}    % E (schwa mobile)
    {/\schwaCons/}    % ə (schwa constans / realization)
    {$\_\_$\{L,N\} (C) + CX}
  \item \deriv
    {/\schwaMobi/}    % E (schwa mobile)
    {/\schwaCons/}    % ə (schwa constans / realization)
    {$\_\_$\{L,N\} (C) + \schwaMobi C(C)}
\end{enumerate}

If we translate the context of these rules into the
vocabulary of \CVCV, we can examine whether its prediction
\tr{of/on} the schwa-vocalisation matches the patterns
described by \citeauthor{issatschenko1974}.

\paragraph{Rule 1}
\Cref{fig:issatschenko voc rule 1}
depicts the situation of rule~1.
While in \ref{fig:issatschenko voc rule 1:L C end}
the liquid or nasal and the consonant enclose an
internal empty Nucleus without lateral actorship,
the vowel following the Schwa in
\ref{fig:issatschenko voc rule 1:L end}
is a \gls{FEN}.

\begin{figure}[h]
  \centering
  \begin{subfigure}{.49\textwidth}
    \centering
    \begin{structure}{}
      \drawVCV{2}
      \V[floating]{\textschwa}
      \C{L/N}
      \emptyV[gov]
      \C{C}
      \fen
      \NoGov{V2}{V1}
    \end{structure}
    \caption{with extra C}
    \label{fig:issatschenko voc rule 1:L C end}
  \end{subfigure}
  \hfill
  \begin{subfigure}{.49\textwidth}
    \centering
    \begin{structure}{}
      \drawVCV{1}
      \V[floating]{\textschwa}
      \C{L/N}
      \fen
      \NoGov{V2}{V1}
    \end{structure}
    \caption{without extra C}
    \label{fig:issatschenko voc rule 1:L end}
  \end{subfigure}
  \caption{Issatschenko vocalisation rule 1 in \CVCV}
  \label{fig:issatschenko voc rule 1}
\end{figure}

\Glspl{FEN} in German principally have the ability
to govern, however the government relationships of
\glspl{FEN} in general \tr{cannot/can not} target
non-empty Nuclei:
\marknote{Move quote to Intro:Gov+Lic?}
\q[p.~643]{scheer2004}{Final empty Nuclei can only
govern Nuclei that are bare of any underlying melody
(floating or attached)} (see \cref{subsec:intro:gov lic}).

Albeit for different reasons, in both variants
the lack of government of $V_1$ results in the
attachment of the floating melody and the Schwa
surfaces; the prediction of \CVCV is consistend with
\citeauthor{issatschenko1974}s rule.

\paragraph{Rules 2 \& 3}
The next two rules show the same characteristics
as rule~1: The Nucleus following the Schwa is either
an internal empty Nucleus or a \gls{FEN}
-- final in respect to a phonological domain
which can, bust doesn't have to be a word.

\paragraph{Rule 4}
In the last rule the situation for words without an
additional $C$ is slightly different.
Here the \ti{/\textsce/} fills the position which
was previously the \gls{FEN} ($V_2$ in \cref{fig:issatschenko voc rule 1:L end}).
\TODO{Quelle finden: in welcher Richtung werden in CVCV
alternierende Vokale ausgewertet?}

\paragraph{Schwa elision rule}
Rules 1--4 describe contexts where a Schwa mobile
gets phonetically expressed. \citeauthor{issatschenko1974}
also lists the following Schwa elision rule which deletes
a Schwa mobile at the start of an affix if the last
vowel of the preceding morpheme is a Schwa constans:

\deriv
  {/\schwaMobi/}    % E (schwa mobile)
  {\O}              % deleted / empty
  {\schwaCons \{L,N\} (C) + $\_\_$ (CC)}%
\footnote{\Textcite{issatschenko1974} actually states the
  following rule:
  \deriv
    {/\schwaMobi/}    % E (schwa mobile)
    {\O}              % deleted / empty
    {\schwaCons \{L,N\} (C) $\_\_$ + \schwaMobi (CC)}\\
  However both his description as well as usage examples
  indicate that the version used above is what he meant.
}

\Citeauthor{issatschenko1974} notes that this rule
applies to contexts where the Schwa constans is underlying
as well as to those where it was generated by one of the
vocalisation rules.
\marknote{Auswirkungen auf CVCV? Wie wird Unterschied dargestellt?}

In \CVCV terms, the Schwa mobile in question is either
the final Nucleus itself (Suffix \ex{-e}) or it is followed
by an empty Nucleus that can not govern it.

\Textcite{scheer2004} isn't clear about the status of
word-final contentful Nuclei:
Either the statement \enquote{the right edge of the word is
  marked by a \acrlong{FEN}} is to be taken
  literally: following the last contentful Nucleus is
  an empty CV-unit, just like the beginning of a word is
  an empty CV-unit.\marknote{source}

Or the contentful Nucleus really is word-final,
i.e. there is no empty Nucleus following it.
Then their situation is unclear.
Are they governed just like \glsplural{FEN}?
Or is this governing of \gls{FEN} restricted to empty
Nuclei, therefore rendering the final contentful Nucleus
ungoverned?

\Citeauthor{scheer2004} seems to indicate that the latter
is the case, when he introduces the special status
of \gls{FEN}: \q[p.~14]{scheer2004}{The latter group is
illustrated for example by Italian where all words must
end in a vowel: final empty Nuclei are not licensed and
hence need to enjoy a melodic existence.}

Since the book develops the \CVCV theory on the basis
of earlier \gls{GP} theories, the special status of
\gls{FEN} is introduced in terms of licensing,
as Standard Government Phonology uses the term
\emph{domain-final Licensing}.
In chapter~7.2 \Citeauthor{scheer2004} argues
that this relation is actually government and calls
it \emph{domain-final Government} thereafter.
Because of this structure, it isn't entirely clear which
statements regard the previous analyses of \gls{GP} and
which translate similarly to \CVCV.
Later on, both the examples and the terminology regard
final Nuclei as empty, the possibility of final Nuclei
with content isn't discussed,
let alone the properties of final Nuclei with a floating
melody, i.e. alternation sites.

If the former is the case, i.e. there is an actually empty
final Nucleus after the final Schwa, \CVCV{}s prediction
is contrary to \citeauthor{issatschenko1974}s analysis.
The context would be similar to the one of rules 1--4,
the Schwa can't be governed and is therefore expected to
surface.

Or the latter is the case and final \emph{empty} Nuclei
actually mean all final Nuclei, the empty case just being
the one that is most interesting for the theory%
\footnote{e.g. because almost the entirety of
  word-/domain-final effects worth examining involve the
  parametrisation of the lateral actorship of empty Nuclei.}.

The prediction made by \CVCV only matches
\Citeauthor{issatschenko1974}s elision rule if those
contentful final Nuclei are governed like their
empty counterparts. In all other cases -- either if
the Suffix contains at least one additional consonant or
because the Schwa is either final but ungoverned or followed
by an actually \emph{empty} final Nucleus --
\CVCV{} predicts the Schwa to be associated and therefore
vocalised.
\marknote{that is, assuming that the suffixes have the same
structure in CVCV. Could there be differences that would
explain this mismatch?}

\noindent\TODO{Vol. 2 states that final filled Nuclei aren't parametrized:\par
  \normalfont\q[242]{scheer2012}{Therefore the phonological (lateral)
  properties of final empty nuclei (but not of final filled nuclei)
  are defined by a parametric choice.}}


\subsubsection{Schwa-initial suffixes}
Comparing the predictions of some exmaple derivations might shed
some light on these parameters.
\lmarknote{\TODO{rewrite}}

When deriving \ex{Handelns}(Gen.Sg), \Citeauthor{issatschenko1974}
first notes that because of the related words \ex{Händler} and \ex{Handlung}
the root must contain a Schwa mobile. To this root the suffixes
\ex{-\schwaMobi{}n} and \ex{-\schwaMobi{}s} are added and the
Schwa mobiles are resolved as follows:
\begin{enumerate}
  % shortcuts for suffixes -en, -es
  \newcommand\sN{\schwaMobi{}n}
  \newcommand\sS{\schwaMobi{}s}
  \item Rule $V_4$: \{(hand\schwaMobi{}l + \sN) + \sS\} $\to$
    \{(hand\schwaCons{}l + \sN) + \sS\}
  \item Rule $T$: \{(hand\schwaCons{}l + \sN) + \sS\} $\to$
    \{hand\schwaCons{}ln + \sS\}
  \item Rule $T$: \{hand\schwaCons{}ln + \sS\} $\to$ \{hand\schwaCons{}lns\}
\end{enumerate}

The analysis in \CVCV depends on the nature of the suffixes:
If they include the Schwa as \citeauthor{issatschenko1974}
assumes, they are vowel-initial and therfore analytic
-- they join the domain of the root and no domain-final
Government is at play here.

For the infinitive suffix <{}-en>, the examples given
in \textcite[p.~305]{scheer2004} -- \ex{geben}, \ex{sagen},
\textellipsis -- indicate that there is an underlying Schwa
and his analysis of the partial regressive assimilation of
German nasals (\ti{[ge:b@n]}--\ti{[ge:b\s{m}]}) reveals the
nature of this vowel: it is an alternating vowel and therefore
a Schwa mobile.

The contrast \ex{das Haus} -- \ex{des Hauses} and
\ex{der Schlauch} -- \ex{des Schlauchs} shows that
the genitive suffix must start with a Schwa mobile as well.

The \CVCV structure must therefore be the following:
\begin{structure}{}
  \drawCV{5}
  \C{h}
  \V{a}
  \C{n}
  \emptyV
  \C{d}
  \V[floating]{\textschwa}
  \C{l}
  \V[floating]{\textschwa}
  \C{n}
  \V[floating]{\textschwa}
  \C{s}
  \fen
\end{structure}
Because the suffixes are vowel-initial and therefore
analytic, they enter the domain of the root.
The Schwas not being realized can't be attributed to
the Government of domain-final Nuclei.

Infrasegmental Government is ruled out because in German
neither the final Nucleus nor any alternation site could
license its head.

Following the Havlík pattern, one would expect $V_3$ and $V_5$
to be vocalised, which is certainly not the case: \ti{*[hand@ln@s]}.

The analysis could be somewhat saved by allowing vowel-initial suffixes
to create a new domain, but still enter the domain of the root.
If the Government of \gls{FEN} does not only apply to empty,
but also to contentful Nuclei, the suffix-Schwas would be governed
and the prediction would match reality:
\begin{structure}{}
  \drawCV{5}
  \C{h}
  \V{a}
  \C{n}
  \emptyV[gov]
  \C{d}
  \V[floating]{\textschwa}
  \C{l}
  \fen[domain]
  \V[floating]{\textschwa}
  \C{n}
  \fen[domain]
  \V[floating]{\textschwa}
  \C{s}
  \fen
\end{structure}


\subsubsection{Alternation sites and Infrasegmental Government}
Another interesting aspect of \CVCV, the interaction between alternation
sites and \gls{IG}, can be observed in \ex{Handlung}:
\begin{structure}{}
  \drawCV{5}
  \C{h}
  \V{a}
  \C{n}
  \emptyV[nogov]
  \C{d}
  \V[floating]{\textschwa}
  \C{l}
  \V{\ti{U}}
  \Ng[nolic]
  \fen
  \Gov{V4}{V3}
\end{structure}
At first, it seems like the empty Nucleus between \ti{[n]} and \ti{[d]}
isn't governed, making the word either \tr{ungrammatical} or forcing
the following Schwa to be vocalised in order to provide Government.
There is another possibility however: \acrlong{IG}.

\medskip\noindent\TODO{actually, according to \cref{subsubsec:IG:Lexicon},
IG \emph{is} stored in the lexicon -\_- }\par\medskip

Before, the Schwa in $V_3$ didn't have
enough lateral actorship to license a domain of \gls{IG},
but with the \ti{[U]} of the suffix, this relation is
now possible. Thus the alternation site isn't governed
by Proper Government but by \gls{IG} and is therefore
able to dispense Government itself.

Because Government and Licensing aren't stored in the lexicon
but are dynamic computations, this change from
Proper Government to \gls{IG} should still abide the
law of structure preservation. Note however that
the configuration of an alternation site being governed
by \gls{IG} wasn't mentioned in the book at all,
however to my knowledge there is nothing in the book indicating
that this is not possible.
\begin{structure}{}
  \drawCV{5}
  \C{h}
  \V{a}
  \C{n}
  \emptyV[gov]
  \C{d}
  \emptyV
  \C{l}
  \V{\ti{U}}
  \Ng[nolic]
  \fen
  \IG{C4}{C3}
  \Lic{V4}{C4}
\end{structure}


\bigskip
\TODO{Widerspruch: \normalfont
Wenn beide Suffixe mit Schwa $\implies$ Suffixe vowel-initial.
\q[p.~631]{scheer2004}{On the other hand, all vowel-initial suffixes are
necessarily nonanalytic under this analysis.
That is, their vowel sits in the root-final empty Nucleus.
Analytic vowel-initial suffixes would create the structure
{[[}...t$\emptyset${]}-V{]}. If it is true that \ti{\l}-vocalisation takes place
domain-finally, then, the \ti{[\l]} should vocalise in intervocalic position,
which of course it does not.}}

Zu Infinitivsuffix \ex{-en} \ti{/-@n/}: S. 305, 601(S.719)


\subsubsection[Synkopierungsregel]{Synkopierungsregel $S$}
\TODO{subsubsection oder paragraph? Vorher waren Regeln paragraph, for the
example section I introduced a new subsubsection}


\subsubsection[TODO]{\TODO{other restrictions/predictions in issatschenko}}
  Adjectives, Verbs?

Verbalnomina: Gelache, Gequake, Geläute, Gesumme,
wenn Verbalstamm auf Schwa+Liquid endet kein -e:
Gejammer, Geknister, Geklapper, Gemurmel, Gerassel