\subsection{Government and Licensing}
\label{subsec:intro:gov lic}

In \CVCV, Government and Licensing are opposing forces:
While Government inhibits the melodic expression of
its target, Licensing makes it stronger.

Where in \cite{scheer2004} both forces are independent
from each other (a constituent can be both governed
and licensed, in which case both forces cancel out),
\cite{scheer2012} refines the model as follows:
\blockquote[\cite{scheer2012}]{
  (68) government over licensing
  
  no constituent can be governed and licensed at the
  same time. In case a constituent can potentially be
  subject to both lateral forces, it will be governed.
}
\TODO{is this relevant to German? maybe just mention but ignore it afterwards}

Each Nucleus always \wordunsure{exhibits}
all available forces: full vowels govern and license
while the ability of schwa and final empty Nuclei to
govern/license is subject to (separate) language-specific
parametrization.
In german, both schwa and final empty Nuclei govern but
do not license.

\includegraphics[width=.5\textwidth]{figures/lic-over-branching-onset.png}

\textquote[{\cite[p.~144]{scheer2012}}]{
  Government and licensing must be counted out because they represent
  phonological computation, rather than representational objects,
  and hence cannot be stored in the lexicon (see §186).
}

\subsection[TODO]{\TODO{}}
\begin{itemize}\color{red}
  \item Domains
  \item analytic vs non-analytic suffixes\par
    \q[p.~630]{scheer2004}{If in addition the suffix \ex{-gar} is analytic
      (see §§24,249,349,630 for discussion of this concept),
      i.e. if it does not enter the domain of the root,
      the underlying structure of the word is \ti{/((sa\l\o)-gar)/}.}\par
    \q[p.~631]{scheer2004}{On the other hand, all vowel-initial suffixes
      are necessarily nonanalytic under this analysis. That is,
      their vowel sits in the root-final empty Nucleus. Analytic
      vowel-initial suffixes would create the structure
      \ti{[(\textellipsis\l\o)-\*{V}]}. If it is true that l-vocalisation takes
      place domain-finally, then, the \ti{[\l]} should vocalise in
      intervocalic position, which of course it does not.}
  \item Havlík: every other alternation site is vocalised,
    counting from the right edge of the sequence\par
    \q[p.~508]{scheer2004}{They may be informally stated as under (306) below.
    \begin{itemize}[widest=(306), leftmargin=*]
    \item[(306)] the two patterns of vowel-zero alternations
      \begin{enumerate}[label=\alph*.]
        \item Havlík\\
          given a chain of alternation sites, vocalise every other one,
          counting from the right margin.
        \item Lower\\
          given a chain of alternation sites, vocalise all of them
          save the last one.
      \end{enumerate}
    \end{itemize}}
  \item \gls{FEN}
  \item concept: relations have a head
  \item long vowels: complement needs to be licensed
  \item \gls{ECP}
  \item FEN cannot govern alternation sites:\\
    \textquote[{\cite[p.~643]{scheer2004}}]{%
      Final empty Nuclei can only govern Nuclei that are bare
      of any underlying melody (floating or attached)}
\end{itemize}