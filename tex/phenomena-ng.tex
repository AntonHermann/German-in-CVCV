\subsection{The velar nasal: \textipa{[N]} vs. \textipa{[Ng]}}
\label{subsec:phenomen:ng}

\TODO{vllt ganze subsection hier rauslassen?
Der Mechanismus wirrd auch bei Parameters $\to$ Schwa (\ref{subsec:params:schwa})
schon beschrieben und das \ti{*[aNlIstIk]}-Gegenbeispiel kann auch in ne Fußnote}

\begin{structure}{\ex{Pingpong} \ti{[pINpON]}}
  \drawCV{5}
  \wordstart
  \C{p}
  \V{\textipa{I}}
  \Ng[nolic]
  \emptyV[gov]
  \C{p}
  \V{\textipa{O}}
  \Ng[nolic]
  \fen
\end{structure}

However, not all occurences of \textipa{[Ng]} can be explained
by \textipa{[g]} being licensed, as \ex{Anglistik} shows:
\begin{structure}{\ex{Anglistik} \ti{*[aNlIstIk]}}
  \drawCV{6}
  \wordstart
  \C{\textipa{P}}
  \V{a}
  \Ng[nolic]
  \emptyV[gov]
  \C{l}
  \V{\textipa{I}}
  \C{s}
  \emptyV[gov]
  \C{t}
  \V{\textipa{I}}
  \C{k}
  \fen
\end{structure}
Because the \textipa{[g]} isn't licensed, the model predicts
the word to be pronounced \textipa{*[aN.lIs.tIk]}, not the
actual \textipa{[aN.glIs.tIk]}.