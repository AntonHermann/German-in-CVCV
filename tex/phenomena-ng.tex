\subsection{\textipa{[N]} vs. \textipa{[Ng]}}

\newcommand\Ng{
  \edef\firstpos{C\theconsonant}
  \C{N}
  \emptyV
  \edef\secondpos{C\theconsonant}
  \C{g}
  \draw (p\secondpos.north) -- (\firstpos.south);
}
\begin{structure}
  \drawCV{6}
  \wordstart
  \C{p}
  \V{\textipa{I}}
  \Ng
  \emptyV
  \C{p}
  \V{\textipa{O}}
  \Ng
  \fen

  \Gov{V2}{V1}
  \NoLic{V4}{C4}
  \Gov{V5}{V4}
  \NoLic{V7}{C7}
\end{structure}

However, not all occurences of \textipa{Ng} can be explained
by \textipa{g} failing to be licensed, as \ex{Anglistik} shows:
\begin{structure}
  \drawCV{7}
  \wordstart
  \C{\textipa{P}}
  \V{a}
  \Ng
  \emptyV
  \C{l}
  \V{\textipa{I}}
  \C{s}
  \emptyV
  \C{t}
  \V{\textipa{I}}
  \C{k}
  \fen

  \Gov{V2}{V1}
  \NoLic{V4}{C4}
  \Gov{V5}{V4}
  \Gov{V7}{V6}
\end{structure}
Because the \textipa{[g]} isn't licensed, the theory predicts
the word to be pronounced \textipa{*[aN.lIs.tIk]}, not the
actual \textipa{[aN.glIs.tIk]}.