\textquote[{\cite[p.~xxxiii]{scheer2012}}]{
  The lateral project and CVCV are introduced at greater length in §§10f.
  The reader should be aware that there are a number of different
  implementations of CVCV (%
    Cyran 2010,
    Szigetvári 2001, 2008,
    Rowicka 1999,
    Polgárdi 2003,
    Scheer 2004%
  ),
  and that the one that is referred to in Part III and elsewhere in
  the book revolves around \cite{scheer2004} and
  the Coda Mirror (Ségéral \& Scheer 2008b, Scheer \& Ziková 2010a).
}

\q[p.~16]{scheerCyran2017}{%
  Of course it should be borne in mind that LTP does not aim to directly
  express static patterns. Rather, it makes predictions for where and how
  phonological processes such as lenition and fortition are likely to occur.
  However, it is also true that some static patterns have resulted from
  historrical processes whose application we do predict.
  Thus, returning to static patterns at the right edge of words,
  we predict that systems with FEN actorship will have robust RT clusters
  -- with the T in strong Coda Mirror position (licensed) -- and very weak
  single final consonants (governed).}