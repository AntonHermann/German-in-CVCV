\begin{deleted}\small
\textquote[{\cite[p.~xxxiii]{scheer2012}}]{%
  The lateral project and CVCV are introduced at greater length in §§10f.
  The reader should be aware that there are a number of different
  implementations of CVCV (%
    Cyran 2010,
    Szigetvári 2001, 2008,
    Rowicka 1999,
    Polgárdi 2003,
    Scheer 2004%
  ),
  and that the one that is referred to in Part III and elsewhere in
  the book revolves around \cite{scheer2004} and
  the Coda Mirror (Ségéral \& Scheer 2008b, Scheer \& Ziková 2010a).
}

\q[p.~16]{scheerCyran2017}{%
  Of course it should be borne in mind that LTP does not aim to directly
  express static patterns. Rather, it makes predictions for where and how
  phonological processes such as lenition and fortition are likely to occur.
  However, it is also true that some static patterns have resulted from
  historrical processes whose application we do predict.
  Thus, returning to static patterns at the right edge of words,
  we predict that systems with FEN actorship will have robust RT clusters
  -- with the T in strong Coda Mirror position (licensed) -- and very weak
  single final consonants (governed).}
\end{deleted}\par\bigskip

The nature of the syllable is a central topic of phonology.
For some languages assumed to be the
fundamental building block of all higher structures,
there are many different models of the internal structure
of syllables.

Most common syllable models assume a tree-shaped structure:
a syllable has several children, the most prominent being
the Nucleus in the center. Arranged before and after the Nucleus
are other segments. Sometimes those segments are \tr{arranged}
in a flat structure, be it
skeletal slots commonly marked by an X \marknote{Cite}
or C-slots \parencite{clements1983}.
Most newer theories however group those segments further:
Segments before the Nucleus belong to an Onset, those after
the Nucleus to a Coda and Nucleus and Coda form an intermediate
structure, the Rhyme. Theories often also employ a mechanism
called \enquote{extrasyllabic consonants}, consonants that
are phonetically expressed, which however seem to be invisible
for rules targeting specific constituents, syllabification
algorithms, syllable weight assignment, \textellipsis\marknote{``etc.''?}.

\begin{deleted}
The Rhyme is analyzed as a separate constituent
because especially to determine the weight of a syllable,
only those constituents in the Rhyme contribute.\marknote{unschöne Formulierung}
\end{deleted}

