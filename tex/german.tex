
\paragraph{Obst}
General \emph{problem} cases seem to be problematic for
\CVCV as well, \ex{Obst} %(\ti{/o:p\o s\o t\o/})
comes to mind here:
The \gls{FEN} can only govern the Nucleus \tr{enclosed} by
\ti{[s]} and \ti{[t]}, the preceding Nucleus can't be governed.

\begin{structure}
  \drawCV{5}
  \wordstart
  \C{\ti{P}}
  \longV{o}
  \C{p}
  \emptyV
  \C{s}
  \emptyV
  \C{t}
  \fen
  \Gov{V1}{V0}
  \NoGov{V4}{V3}
  \Gov{V5}{V4}
\end{structure}

\gls{IG} can't save the day either:
While \ti{[s]} and \ti{[t]} do form a valid context for \gls{IG},
which would then be able to govern the preceding Nucleus,
this fails because \gls{IG} needs licensing support from a following
Nucleus, but the \gls{FEN} in german can just govern, not license.

The only other possibility is to resort to somewhat of a loophole:
\ti{[st]} could be analyzed as a \emph{pseudosuffix}.
In this case, there is a domain boundary between \ti{[p]}
and \ti{[s]}, making \emph{V3} %TODO
domain-final and therefore governed.


\paragraph{Overgeneration}
If \glspl{FEN} do govern, too little restriction on word-final consonant
clusters:

\ti{*[Sa:kf]}
\begin{structure}
  \drawCV{4}
  \wordstart
  \C{\ti{S}}
  \longV{a}
  \C{k}
  \emptyV
  \C{f}
  \fen
  \Gov{V1}{V0}
  \Gov{V4}{V3}
\end{structure}

Maybe it's not just Government but consonant clusters are restricted
in general? How do they differ from "anything goes initially" languages?
Source?