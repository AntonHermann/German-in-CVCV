
\paragraph{Obst}
General problem cases seem to be problematic for
\CVCV as well, \ex{Obst} %(\ti{/o:p\o s\o t\o/})
comes to mind here:
The \gls{FEN} can only govern the Nucleus enclosed by
\ti{[s]} and \ti{[t]}, the preceding Nucleus can't be governed.

\begin{structure}{\ex{Obst} \ti{[Po:pst]}}
  \drawCV{4}
  \wordstart
  \C{\ti{P}}
  \longV{o}
  \C{p}
  \emptyV[nogov]
  \C{s}
  \emptyV[gov]
  \C{t}
  \fen
\end{structure}

\gls{IG} can't save the day either:
While \ti{[s]} and \ti{[t]} do form a valid context for \gls{IG},
which would then be able to govern the preceding Nucleus,
this fails because \gls{IG} needs licensing support from a following
Nucleus, but the \gls{FEN} in german can just govern, not license.

Another possibility would be to resort to somewhat of a loophole:
\ti{[st]} could be analyzed as a \emph{pseudosuffix}.
In this case, there is a domain boundary between \ti{[p]}
and \ti{[s]}, making $V_4$
domain-final and therefore governed.

\Cite{scheer2004} however resorts to another interpretation:

\TODO{\q[p.~424]{scheer2004}{§350 4.2.2.
  You can get a coronal for free in Germanic}\normalfont

  \q[p.~425]{scheer2004}{Now it does not take long to see that these
    supernumerary consonants are coronal obstruents in all cases:
    Herb<st>, Mark<t>, Arz<t>.
    
    In fact, we face a stable pan-Germanic pattern.
    It is notorious that whatever the "regular" phonotactics of a
    Germanic language is, "you can get a coronal for free" on top
    of that at the end of words.
    This distributional fact is well studied: Hall (1992:110ss) and
    Wiese (1991) report on German,
    Goldsmith (1990:140ss), Myers (1987) and many others on English,
    Ewen \& Hulst (2001:136ss) on Dutch.

    What should be the phonological attitude towards these
    extra coronals? Obviously, we face but one particular effect of
    the well-known peculiarity of coronality,
    which appears on many other occasions.
    There is a large literature on the special status of coronals:
    Paradis \& Prunet (1991) collect some of the evidence.

    Hence, it seems that the existence of extra coronals at the right
    edge of words has got nothing to do with the phonotactics of Germanic.
    Restrictions on clusters disregard additional coronals.
    If this is true, we are left with little evidence for right edge
    extrasyllabicity. It does not matter whether we wish to count
    sonorants into the class of supernumerary coronals or not
    (in words such as \ex{Harn} \ti{[harn]} "urine").
    In any event, the maximal number of extrasyllabic items that may
    occur at the end of words is one.}

  \q[p.~426]{scheer2004}{I therefore take this working hypothesis for
  granted: word-finally, natural language produces one extrasyllabic
  consonant at most.}
}


\paragraph{\TODO{sanft}}


\paragraph{Overgeneration}
If \glspl{FEN} do govern, too little restriction seems to be placed on
word-final consonant clusters just by \CVCV itself as any combination of
two consonants can be word-final -- the \gls{FEN} governs
the enclosed empty Nucleus either way:

\begin{structure}{\ti{*[Sa:kf]}}
  \drawCV{3}
  \wordstart
  \C{\ti{S}}
  \longV{a}
  \C{k}
  \emptyV[gov]
  \C{f}
  \fen
\end{structure}

Note that \cite{scheer2004} is not entirely clear about this issue:
On the one hand, he states that
\q[p.~580]{scheer2004}{consonant clusters need a license in order to exist.
If they fail to be licensed by a vocalic source, a reaction is expected.}
and this cluster clearly fails to get licensed.

\q[p.~178]{scheer2004}{initial \ctx{*\#RT} clusters are ruled out in typical
Indo-European languages because their putative head R fails to be licensed
to govern by its Nucleus, which is empty.}

On the other hand, there doesn't seem to be a clear rule enforcing this.
All examples given are in regard to branching Onsets and therefore \gls{IG},
where the need of a licensing vowel is made explicit.

Further, the \ctx{*\#RT} example is explained differently later on:
The well-formedness of initial \ctx{\#RT} clusters is determined by the existence
of an initial empty CV-unit:
In languages where this CV-unit is present, the empty Nucleus between the two consonants
has to govern the initial empty Nucleus, which is only possible if there is
an \gls{IG} relation between them -- therefore ruling out \ctx{\#RT} sequences.
If this initial CV-unit is absent, there are no restrictions on the initial consonants.
Both \ctx{\#TR} and \ctx{\#RT} are well-formed as long as the enclosed empty Nucleus
is properly governed by a following contentful Nucleus.
There does not seem to be any reason that this wouldn't apply equally to word-final
consonant clusters. On the contrary, he even states the following:
\q[p.486]{scheer2004}{As everywhere else in the grammar when an empty Nucleus is
silenced by Government, the distribution of the flanking consonants is free.}